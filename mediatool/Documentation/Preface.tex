

\section{Preface}


\subsection*{General information}
This documentation contains the mediatool specifications.
The specifications are regarded as V0.40. Last revision is from Sun Jun 15 12:48:45 CEST 1997.


\subsection*{Topics not yet adressed in this document}
There are a few things, which are still open for discussion. In fact, discussion about
theese things are very welcome. Even if I have have an opinion, what might be right,
I am aware I could be wrong.

Please do skip this section for the first time you read this document. You will need
some basic information to understand the meaning of some terms, which are used here
(e.g. ``master'', ``slave'' or ``chunk'').

\begin{itemize}
\item	Playlists: These will be handled by the library code in the future. The master fills
	the playlist using library functions. The advantage of playlist handling in the master
	is, that it has to be programmed only once, and not again and again in
	each slave. Furthermore, the master can have a playlist with different media types:
	For example, there could be a playlist with your all-time-favourites: Theese could
	be ``greatmidi1.mid'', ``greatmidi2.mid'', ``track3.cda'', ``greatmod4.mod'' and
	``greatsid5.psid''. (*.mid = Midi music, *.cda = CD Audio,
	*.mod = Soundtracker module, *.psid = C64 Sid music).
\item	Text informations from slave to master (There is no use for the other direction,
	as far as I can see). I am planning to do a big protocol revision, where this topic
	will be adressed. The format of the ``text chunk'' is not fully defined, but
	it will most likely have a ``catergory'' field (e.g. MD\_CAT\_INSTUMENTNAME), one
	or more extra fields for storing extra informations (e.g. instrument number) and
	the text field (e.g.  the instrument name).
\item	\ldots
\end{itemize}


